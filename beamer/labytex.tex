\documentclass{beamer}
% francification de LaTeX
\usepackage[utf8]{inputenc}
\usepackage[french]{babel}
% imagination
\usepackage{tikz}
% options de beamer
\usetheme{Boadilla}
\title{Projet \textit{Labyrinthe}}
\subtitle{Algorithmes et Structures de Donnée}
\author{Juan-Carlos Barros et Daniel Kessler}
% et c'est parti
\begin{document}

\begin{frame}
  \titlepage
\end{frame}

\begin{frame}
%  \frametitle{}
  % Formation \textit{GymInf},
  \onslide<-2>{Cours d'\textit{\textcolor{blue}{Algorithmes} et
      \textcolor{green}{Structures de donnée}}}
  \par\bigskip
  \onslide<2->{Projet \textit{Labyrinthe}}
  \begin{itemize}
  \item<3->\textcolor{blue}{Algorithme:} \onslide<4->{A*}
  \item<3->\textcolor{green}{Structure de Donnée:} \onslide<5>{Priority Queue}
  \end{itemize}
\end{frame}

\begin{frame}
  \frametitle{Table des matières}
  \tableofcontents
\end{frame}

\section{Problème(s)}
\begin{frame}
  \frametitle{Un labyrinthe, plusieurs problèmes}
  \begin{itemize}
  \item Cherche-t-on un chemin quelconque?
    \begin{itemize}
    \item<2-> Oui, on ne traversera le labyrinthe qu'une seule fois.
    \item<2-> \textbf<4->{On veut le chemin le plus court}, pour peut-être le réutiliser.
    \end{itemize}
  \item Connait-on les coordonnées de la sortie dès le départ?
    \begin{itemize}
    \item<4-> \textbf{oui}
    %\item<3-> non
    \end{itemize}
  \end{itemize}
\end{frame}
\begin{frame}
  \frametitle{Un problème, plusieurs solutions}
  \begin{itemize}
  \item Breadth-First
  \item Dijkstra
  \item A*
  \end{itemize}
\end{frame}

\section{Test de pseudo-code}
\begin{frame}
  \frametitle{Effets simples dans beamer}
  \begin{block}{ceci est un bloc}
    On peut pseudo-inclure du code.
  \end{block}
  \begin{definition}
    Le pseudo-code est un outil de communication entre humains.
  \end{definition}
  \begin{example}
    Ce qui suit est en fait du vrai code: un outil de communication humain-machine.
  \end{example}
  \begin{semiverbatim}
    print("Hello, world!")
  \end{semiverbatim}
  \begin{alertblock}{Attention!}
    Ce slide ne doit pas être gardé dans la vraie présentation!
  \end{alertblock}
\end{frame}

\section{Exemple}
\begin{frame}
\frametitle{Labyrinthe de démonstration}

\begin{tikzpicture}[scale=.5]
  % Grid
  \draw[lightgray!20] (0,0) grid (8,8);
  
  % Puzzle
  \draw[line width=3pt,
  cap=round,
  rounded corners=1pt,
  draw=black!75] (-0.5,-0.5) -| (8.5,8.5)
  (7.5,8.5)-|(-0.5,0.5) -- (0.5,0.5) (-0.5,2.5)-|(0.5,4.5)
  (0.5,5.5)|-(4.5,7.5)|-(3.5,3.5)|-(2.5,2.5)|-(3.5,1.5)
  (8.5,0.5)--(6.5,0.5)
  (5.5,0.5)-|(3.5,-0.5)
  (3.5,0.5)-|(1.5,3.5)-|(2.5,5.5)
  (2.5,4.5)-|(3.5,6.5)-|(1.5,4.5)
  (7.5,8.5)|-(5.5,6.5)--(5.5,7.5)
  (6.5,7.5)--(6.5,8.5)
  (8.5,3.5)-|(6.5,4.5)-|(7.5,5.5)
  (6.5,5.5)-|(5.5,2.5)--(4.5,2.5)
  (5.5,2.5)-|(7.5,1.5)--(4.5,1.5)
  (5.5,4.5)--(4.5,4.5)
  (1.5,1.5)--(0.5,1.5)
  (1.5,7.5)--(1.5,8.5);
  
  % Start and End Points
  \onslide<2>{\draw[-latex,line width=3pt,red] (-1,0)--(0,0);}
  \onslide<3>{\draw[-latex,line width=3pt,red] (8,8) -- (8,9);}
\end{tikzpicture}
\end{frame}
\end{document}