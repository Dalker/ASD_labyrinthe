\documentclass[]{beamer}
% \documentclass[handout]{beamer}
% francification de LaTeX
\usepackage[utf8]{inputenc}
\usepackage[french]{babel}
% urls cachés
\usepackage{hyperref}
% imagination
\usepackage{tikz}
% options de beamer
\usetheme{Boadilla}
\title{Projet \textit{Labyrinthe}}
\subtitle{Algorithmes et Structures de Données}
\author{Juan-Carlos Barros et Daniel Kessler}
% des compteurs pour l'importation d'images
\usepackage{forloop}
\newcounter{onlynumber}
\newcounter{pngnumber}
% code
\usepackage{listingsutf8}
\definecolor{lstcolor}{rgb}{0.9,0.95,0.95}
\definecolor{lstcommentcolor}{rgb}{0.,0.2,0.}
\lstset{
  frameround=tttt,
  %autogobble,
  frame=single,
  backgroundcolor=\color{lstcolor},
  % extendedchars=true,
  % basicstyle=\ttfamily\small,
  keywordstyle=\bfseries\color{blue},
  identifierstyle=\bfseries\color{red},
  stringstyle=\bfseries\color{orange},
  commentstyle=\color{lstcommentcolor},
  language=Python,
  keepspaces=True,
  basicstyle=\fontfamily{pcr}\selectfont\small, % monospace it for copypasting
  upquote=true,
  columns=flexible,
  showstringspaces=False,
  literate={é}{{\'e}}1
}
% et c'est parti
\begin{document}

\begin{frame}
  \titlepage
\end{frame}

\begin{frame}
%  \frametitle{}
  % Formation \textit{GymInf},
  Cours d'\textit{\textcolor<1>{blue}{Algorithmes}}
  et \textit{\textcolor<1>{green}{Structures de données}}
  \par\bigskip
  \onslide<2->{Projet \textit{Labyrinthe}}\par
  \begin{minipage}{.65\linewidth}
  \begin{itemize}
  \item<2->\textcolor{blue}{Algorithme}\par\onslide<3->{A*}
  \item<2->\textcolor{green}{Structures de Données}\par\onslide<4->{Objet "labyrinthe"}
  \par\onslide<5->{Priority Queue}
%  \par\onslide<6->{Tableau des parents}
  \end{itemize}
  \end{minipage}
  \begin{minipage}{.3\linewidth}
  \onslide<2->{\includegraphics[width=\linewidth]{../diapos/V2bis/page1.png}}    
  \end{minipage}
\end{frame} % Titre

\begin{frame}
  \frametitle{Table des matières}
  \tableofcontents
\end{frame} % toc

\section{Quel algorithme pour résoudre quel problème?}
\subsection{Choix du problème}
\begin{frame} 
  \frametitle{Un labyrinthe, plusieurs problèmes}
  \begin{itemize}
  \item<1-> Cherche-t-on un chemin quelconque?
    \begin{itemize}
    \item<2-> Oui, on ne traversera le labyrinthe qu'une seule fois.
    \item<2-> \textbf<3->{Non, on veut le chemin le plus court},
      pour peut-être le réutiliser.
    \end{itemize}
  \item<4-> Connait-on les coordonnées de la sortie dès le départ?
    \begin{itemize}
    \item<5-> \textbf<6->{Oui, et cette information pourra nous aider.}
    \item<5-> Non, le lieu de la sortie fait partie des inconnues.
    \end{itemize}
  \end{itemize}
% début du doublon avec partie rajoutée spécifiquement sur les structures de données employées dans notre projet
  \bigskip
  \onslide<7->{Pour nous, un labyrinthe sera encapsulé dans un objet pouvant
  être consulté afin d'obtenir:}
  \begin{itemize}
  \item<8-> les coordonnées de l'entrée
  \item<9-> les coordonnées de la sortie
  \item<10-> une méthode pour demander s'il est possible de traverser une
    cellule à des coordonnées données
  \end{itemize}
% fin du doublon
\end{frame} % choix du problème

\subsection{Choix de l'Algorithme}
\begin{frame}
  \frametitle{Un problème, plusieurs solutions}
  \begin{tabular}{p{.65\textwidth}p{.35\textwidth}}
    %\vspace{-15ex}
  \begin{itemize}
  \item<1-> Breadth-First Search
    \begin{itemize}
    \item garantit de trouver une solution si elle existe
    \item solution optimale si tous les pas sont égaux (même coût)
    \end{itemize}
  \item<2-> Dijkstra
    \begin{itemize}
    \item choisit où explorer selon les distances déjà parcourues
    \item garantit de trouver le plus court chemin (en tenant compte des coûts)
    \end{itemize}    
  \item<3-> A*
    \begin{itemize}
    \item nécessite de connaître les coordonnées de la sortie
    \item choisit où explorer selon les distances déjà parcourues et une estimation de la distance à la sortie
    \end{itemize}
  \end{itemize}
  & \par\smallskip\vspace{-2ex}\phantom+
    \includegraphics[width=.6\linewidth]{../diapos/V1/page1.png}\par\phantom+
    \onslide<2->{\includegraphics[width=.6\linewidth]{../diapos/V1/graphe.png}}
  \end{tabular}
\end{frame}

\section{Algorithme A*}
\subsection{Pseudo-Code}
\begin{frame}
  \frametitle{Algorithme A*}
  \begin{block}{pseudo-code}
    Démarrer une file d'attente avec la cellule de départ $D$, une liste de
    prédécesseurs avec $\{D:Nil\}$ et une liste de coûts d'accès avec
    $\{D:0\}$.
    \medskip
    
    \onslide<2->{Tant que la file d'attente n'est pas vide,}
    \begin{itemize}
    \item<3-> extraire (pop) la cellule prioritaire $C$ de la file d'attente
    \item<4-> si $C$ est la cellule d'arrivée $A$, retourner le chemin qui y amène (via backtracking sur les prédécesseurs)
    \item<5-> sinon, pour chaque voisin $V$ de $C$ qui n'est pas déjà accessible à moindre coût
      \begin{itemize}
      \item<6-> mémoriser le prédécesseur de $V$ (soit la cellule $C$) et le
        coût d'accès à $V$ (soit un de plus que le coût d'accès à $C$)
      \item<8-> insérer (push) $V$ dans la file d'attente
      \end{itemize}
    \end{itemize}    
  \end{block}
\end{frame}

\subsection{Heuristique et Priorité}
\begin{frame}
  \frametitle{Heuristique et Priorité}
  La priorité d'une cellule $C$ en attente est le coût total (estimé) d'un chemin passant par cette cellule.
  \smallskip
  priorité = coût\_réel ($D \rightarrow C$) + coût\_estimé ($C\rightarrow A$)


  \onslide<2->{La distance restante depuis une cellule jusqu'à l'arrivée doit être estimée
  \textit{sans jamais la surestimer}. $ \rightarrow $ choix d'heuristique!}
  \medskip
  \begin{itemize}
  \item<3->{La \textbf{distance de Manhattan} $|\Delta x| + |\Delta y|$ est un bon estimateur 
  si les mouvements permis sont horizontaux et verticaux.}
  \par
  \item<4->{Une \textbf{heuristique nulle} ramène A* à l'algorithme de Dijkstra 
  (ou Breadth-First Search si on a un coût identique pour chaque mouvement).}
  \end{itemize}
\end{frame}

\subsection{Idée de preuve}
\begin{frame}
  \frametitle{Preuve de l'algorithme (grandes lignes)}
  L'heuristique $h(C)$ qui estime le chemin restant depuis une cellule $C$
  doit satisfaire deux conditions.
  \begin{enumerate}
  \item<2-> \textbf{Monotonicité}: sorte d'inégalité triangulaire faible
    $h(C_1) \leq r(C_1,C_2) + h(C_2)$ où $C_1, C_2$ sont deux cellules,
    $r(C_1, C_2)$ est la distance réelle entre elles.
% Une image serait pratique pour la triangulation etc.
    \par\smallskip
    \onslide<2->{Cette propriété garantit de trouver la sortie, sans se
      ``perdre'' dans des boucles éventuelles.}
  \item<3-> \textbf{Admissibilité: l'heuristique ne surestime jamais une distance}
    % "admissibilité" ??? Plutôt "sous-estimation", ou qqch dans ce genre?
    % toutes les réfs parlent d'admissibilité...
   % mais ça fait doublon avec "doit satisfaire deux conditions", donc les deux sont des "admissibilités", non?
    \par\smallskip
    \onslide<4->{Cela garantit qu'il n'y a pas de chemin plus court que celui
      trouvé.}\par\smallskip \onslide<5->{Par contradiction: }\onslide<6->{s'il
      y en avait un, il aurait été estimé correctement ou sous-estimé, et serait
      donc prioritaire par rapport à un chemin complet trop long, vu
      qu'\textbf{un chemin complet a une priorité calculée en coût réel. (coût
        heuristique de cellule A = 0)}}
  \end{enumerate}
\end{frame}

\subsection{Exemple de résolution}
\begin{frame}
  \frametitle{Exemple de résolution}
  \begin{minipage}{.4\textwidth}    
    \only<1>{\includegraphics[width=\linewidth]{../diapos/V2bis/page1.png}}
    \only<2>{\includegraphics[width=\linewidth]{../diapos/V2bis/page2.png}} % 2 ou 4
    \only<3>{\includegraphics[width=\linewidth]{../diapos/V2bis/page5.png}} % 5 ou 6
    \only<4>{\includegraphics[width=\linewidth]{../diapos/V2/page1.png}}
    \setcounter{pngnumber}{3}
    \forloop{onlynumber}{5}{\value{onlynumber} < 66}{
      % \only<\arabic{onlynumber}>{only \theonlynumber, png \thepngnumber}\par
      \only<\arabic{onlynumber}>{\includegraphics[width=\linewidth]{../diapos/V2/page\arabic{pngnumber}.png}}
      \stepcounter{pngnumber}
    }
  \end{minipage}
  \quad
  \begin{minipage}{.55\textwidth}
    \only<1>{\textbf{Objectif}\par
      Trouver le chemin le plus court entre le \textbf{D}épart et
      l'\textbf{A}rrivée} \only<2>{Chaque noeud est à une distance réelle du
      départ, qui sera découverte en cours de route.}  \only<3>{Estimation de la
      distance réelle de l'arrivée pour chaque cellule (Heuristique
      ``Manhattan'').}  \only<4>{Le départ est mis en file d'attente, avec une
      priorité 0.}  \only<5>{Le seul voisin est évalué:
      \begin{itemize}
      \item coût réel pour y accéder: \textcolor{green}1
      \item coût heuristique pour la suite: \textcolor{red}{17}
      \item coût heuristique total (priorité): 18
      \end{itemize}
    }
    \only<6-43>{
      Légende:
      \begin{itemize}
      \item\textcolor{green}{coût réel jusqu'ici}
      \item\textcolor{red}{coût heuristique pour la suite}
      \item {coût heuristique total}
      \end{itemize}
      \medskip
      L'algorithme poursuit son chemin.}
    \par\medskip
    \only<7-8>{Parfois deux choix ont la même priorité, le choix est arbitraire.}
    \only<18-29>{Parfois la priorité participe vraiment au choix.}
    \only<34,38>{Parfois un nouveau chemin améliore l'accès à une même cellule.}
    \only<43->{Un chemin vers la sortie a été trouvé!}
    \onslide<44->{Backtracking pour reconstituer le chemin}
  \end{minipage}
\end{frame}

\subsection{Complexité abstraite (sans structure de données)}
\begin{frame}
  \frametitle{Complexité abstraite de l'algorithme - 1}
  Le pire des cas sera réalisé pour un labyrinthe dont le meilleur chemin revient
  souvent en arrière (s'éloigne de l'arrivée). Dans le cas extrême, aucun gain n'est
  réalisé par rapport à l'algorithme de Dijsktra (ou ``heuristique nulle'').
  \par\medskip
  \onslide<2->{Cependant, le ``worst case'' ne rend pas justice à A* dont le
    but est justement d'éviter la plupart du temps le ``worst case'' avec un bon
    choix d'heuristique.}
\end{frame}

\begin{frame}
  \frametitle{Complexité abstraite de l'algorithme - 2}
  \onslide<1->\begin{itemize}
  \item ``worst case'' $\rightarrow$ visite toutes les N cellules. 
   \onslide<2->\begin{itemize} 
   \item \textbf {Pop} pour chaque cellule visitée, \textbf {push} de chaque voisin atteignable (si pas encore pushé). 
   \onslide<3->\item Sinon, \textbf {raffraichissement} éventuel de la liste d'attente 
   \end{itemize}
  \onslide<4->\item Le nombre de \textbf {pop} et de \textbf {push} sera donc au maximum N.
  \onslide<5->\item Le nombre de \textbf {raffraichissements} sera au maximum de $A$ 
  où $A$ = nombre de mouvements possibles.
    \onslide<6->\begin{itemize} 
    \item $O({A}) \sim O({N})$ vu que chaque cellule possède au maximum 4 voisins atteignables.
    \end{itemize}
   \end{itemize}
  \onslide<7->\begin{block}{complexité algorithmique}
  $N \cdot (pop + push) + A \cdot (raffraichissement)$
  \end{block}
  \onslide<8-> L'implémentation de la liste d'attente est cruciale.  La complexité de \textbf {push}, \textbf {pop} et \textbf {raffraichissement} dépendent en effet du type de file mise en oeuvre!
\end{frame}

\section{Structures de données}
\subsection{Représentation des Labyrinthes}
 \begin{frame}
 \frametitle{Structure de nos labyrinthes}
  \begin{minipage}{.3\linewidth}
  \textit{OBJET} "Labyrinthe"
  \includegraphics[width=\linewidth]{../diapos/V2bis/page1.png}

 \end{minipage}
 \begin{minipage}{.65\linewidth}
   \begin{itemize}
   \onslide<2->\item basé sur une matrice de booléens n x m (lignes x colonnes)
   \onslide<3->\begin{itemize}\item chaque cellule de la matrice correspond à une case du labyrinthe
   \onslide<4->\item True = permis (case blanche), 
   \onslide<5->\item False = obstacle (case noire)
   \end{itemize}
   \onslide<6->\item Solveur peut questionner l'objet :
   \onslide<7->\begin{itemize}\item Case (x,y) atteignable ?
   \onslide<8->\item Coordonnées du départ $D$
   \onslide<9->\item Coordonnées de l'arrivée $A$
   \end{itemize}
  \end{itemize}

 \end{minipage}
\end{frame}


% Ce serait bien de parler de la structure de donnée employée pour les parents (dictionnaire?) 
% \subsection{Tableau des parents}

\subsection{``Priority Queue''}
\begin{frame}
  \frametitle{File d'attente: ``Priority Queue''}
  \begin{itemize}
  \item Structure permettant ``push'' avec priorité et ``pop'' rapide de
    l'élément prioritaire
  \item Implémentation en Python en tant que \textbf{binary heap} avec le module \textit{heapq}
  \item Dans cette implémentation, vérifier si vide en $O(1)$, ``push'' et
    ``pop'' en $O\left(\log(n)\right)$ où $n$ est le nombre d'objets en attente\footnote{cf. https://www.cs.princeton.edu/~wayne/kleinberg-tardos/pdf/BinomialHeaps.pdf}
  % \item Un élément est un tuple (heuristique, numéro, contenu), ainsi en cas
  %   d'heuristique égale, le numéro le plus bas (insertion plus ancienne) donne
  %   la priorité
  \end{itemize}
\end{frame}
% \begin{frame}
%   \frametitle{Priority Queue pour A*}
%   \begin{itemize}
%   \item<1-> La priorité d'un noeud est le \textbf{coût heuristique} pour aller de
%     l'entrée à la sortie en passant par ce noeud.
%   \item<2-> Un élément de la queue a comme attributs l'identité du noeud, le
%     \textbf{coût réel} pour y accéder et son prédécesseur
%   \item<3-> La queue est initialisée avec le noeud de départ et un coût nul.
%   \end{itemize}
% \end{frame}

\subsection{Complexité en fonction de la structure de donnée utilisée}
\begin{frame}
  \frametitle{Complexité - structure de données}
  \begin{block}{complexité algorithmique}
  $N \cdot (pop + push) + A \cdot (raffraichissement)$
  \end{block}
  \onslide<2->\begin{itemize}
  \item simple tableau:
  \textbf {push} et \textbf {raffraichissement} en $O(1)$, \textbf {pop} en $O(N)$
  \begin{block}{complexité totale avec simple tableau}
  $N \cdot (O(N) + O(1)) + A \cdot O(1) = O(N^2)$. 
  \end{block}
  \onslide<3->\item queue prioritaire binaire:
  \textbf {push} et \textbf {raffraichissement} en $O(\log(N))$ ,  \textbf {pop} en $O(1)$.
  \begin{block}{complexité totale avec queue prioritaire binaire}
  $N \cdot (O(1) + O(\log(N))) + A \cdot O(\log(N)) = O(N \cdot \log(N))$
  \end{block}
  \end{itemize}
\end{frame}

\section{Tests avec Python}
\defverbatim[colored]\code{
   marge = QueuePrioritaire(grid.start)\\
   cout\_reel = \{grid.start: 0\}\\
   parent = \{grid.start: None\}\\
\par\par
    while True:
        noeud\_courant = marge.pop()
        if noeud\_courant is None:
            raise ValueError(``A*: la grille fournie n'a pas de solution'')
        if noeud\_courant == grid.out:
            break  \# on a trouvé un chemin optimal vers la sortie
        for direction in ((0, 1), (1, 0), (0, -1), (-1, 0)):
            \# calculer coordonnées d'un voisin potentiel
            voisin = tuple([noeud\_courant[i] + direction[i] for i in (0, 1)])
            if voisin not in grid:
                continue  \# on ne peut pas accéder à ces coordonnées
            cout\_voisin = cout\_reel[noeud\_courant] + 1
            if voisin in cout\_reel and cout\_voisin >= cout\_reel[voisin]:
                continue  \# on a un meilleur chemin pour arriver à ce voisin
            \# on est arrivé jusqu'ici: ajouter le voisin à la marge
            cout\_reel[voisin] = cout\_voisin
            parent[voisin] = noeud\_courant
            marge.insert(cout\_voisin + distance(voisin, grid.out), voisin)
}
\begin{frame}[fragile]
  \frametitle{Tests avec Python}
  \begin{lstlisting}
marge = QueuePrioritaire(grid.start)
cout_reel = {grid.start: 0}
parent = {grid.start: None}

while True:
    noeud_courant = marge.pop()
    if noeud_courant is None:
        raise ValueError(``la grille n'a pas de solution'')
    if noeud_courant == grid.out:
        break  # chemin optimal trouvé
    # ... traiter noeud courant
\end{lstlisting}
la suite dans: % \url{https://github.com/Dalker/ASD_labyrinthe/}
\href{https://github.com/Dalker/ASD\_labyrinthe/tree/main/implementation}{https://github.com/Dalker/ASD\_labyrinthe/}
\begin{enumerate}
\item cloner le dossier \texttt{implementation}
\item exécuter \texttt{visual\_test.py} et \texttt{time\_test.py}
\end{enumerate}
\end{frame}
\begin{frame}
  \frametitle{Tests avec Python}
  \includegraphics[width=.9\linewidth]{manhattan_vs_null.png}
\end{frame}

\begin{frame}
  \frametitle{Tests avec Python}
  \begin{semiverbatim}
    * Comparaison heuristique nulle vs Manhattan distance *
    
    solveur1 = heuristique 0, solveur2 = heuristique Manhattan
    
    30x30  : generate=0.1019s solve1=0.0060s solve2=0.0022s
    
    40x40  : generate=0.1814s solve1=0.0096s solve2=0.0086s
    
    50x50  : generate=0.5002s solve1=0.0228s solve2=0.0204s
    
    60x60  : generate=0.8701s solve1=0.0280s solve2=0.0199s
    
    70x70  : generate=1.0461s solve1=0.0451s solve2=0.0391s
    
    80x80  : generate=1.4218s solve1=0.0563s solve2=0.0423s
  \end{semiverbatim}
\end{frame}

\section{Conclusion}
\subsection{Ants}
\begin{frame}
  \frametitle{Conclusion - Autres "Path Finding" Ants}
  «ACO» : Ant Colony Optimisation
   \onslide<2->\begin{itemize}
   \item imite la nature avec dépôt de phéromones
   \onslide<3->\item évaporation des phéromones $\rightarrow$ disparition des chemins longs
   \onslide<4->\item Avantage: adaptable si le labyrinthe évolue au cours du temps!
   \onslide<5->\item Désavantage: necessite un grand nombre d'individus
   \end{itemize}
  \begin{minipage}{.45\linewidth}
   \includegraphics[width=\linewidth]{Aco_branches.svg.png}
  \end{minipage}
  \begin{minipage}{.5\linewidth}
   \begin{itemize}
	\item 1. 1ère fourmi revient de l'arrivée (nourriture) vers le départ (nid) en laissant une piste de phéromone
   \item 2. Autres fourmis parcourent des chemins parfois légèrement différents
   \item 3. Les chemins plus longs disparaissent à cause de l'évaporation plus forte des phéromones.
   \end{itemize}
  \end{minipage}
  Johann Dréo — https://commons.wikimedia.org/w/index.php?curid=815564
\end{frame}

\subsection{Robots}
\begin{frame}
  \frametitle{Conclusion - Autres "Path Finding" Robots}
  Robots aspirateurs
  \onslide<2->\begin{itemize} \item But un peu différent car doivent passer partout 1 seule fois
  \onslide<3->\item mais doivent aussi trouver un chemin d'un point A vers un point B
  \onslide<4->\item utilisent souvent des algorithmes évolutionnistes:
  \onslide<5->\item Gènes = paramètres à optimiser (chemins) 
  \onslide<6->\item Transmission de gènes, via reproduction "d'individus"
  \onslide<7->\item Mélanges + mutations $\rightarrow$ calcul d'adaptabilité
  \onslide<8->\item Elimination statistique des individus moins performants
  \onslide<9->\item Avantage: Adaptable si le labyrinthe se modifie
  \onslide<10->\item Désavantage: Besoin de beaucoup d'individus et beaucoup de générations 
   pour parvenir à un bon résultat  
  \end{itemize}
\end{frame}

\section{Références}
\begin{frame}
  \frametitle{Références}
  \begin{itemize}
  \item Liste des sources consultées:
    \url{https://github.com/Dalker/ASD_labyrinthe/wiki/Sources}
  \item Notre implémentation en Python, avec tests temporels et ``tests
    visuels'' (animations):
    
    \url{https://github.com/Dalker/ASD_labyrinthe/tree/main/implementation}
  \end{itemize}
\end{frame}
\end{document}